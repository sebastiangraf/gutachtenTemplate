% Dokumenteinstellungen und Anpassungen
\documentclass[chapterprefix=false, 12pt, a4paper, oneside, parskip=half, listof=totoc, bibliography=totoc, numbers=noendperiod]{article}

%\renewcommand{\familydefault}{\sfdefault}
%\usepackage{helvet}


%Anpassung der Seitenränder (Standard bottom ca. 52mm abzüglich von ca. 4mm für die nach oben rechts gewanderte Seitenzahl)
\usepackage[bottom=35mm,left=25mm,right=25mm]{geometry}

\usepackage{csquotes}
\usepackage[round,sort&compress, authoryear]{natbib}
\bibliographystyle{plainnat}

%Tweaks für scrbook
\usepackage{scrhack}

\usepackage{tabularx} % in the preamble

%Blindtext
\usepackage{blindtext}

%Erlaubt unter anderem Umbrüche captions
\usepackage{caption}

%Stichwortverzeichnis
\usepackage{imakeidx}

%Kompakte Listen
\usepackage{paralist}

%Zitate besser formatieren und darstellen
\usepackage{epigraph}

%Glossar, Stichwortverzeichnis (Akronyme werden als eigene Liste aufgeführt)
\usepackage[toc, acronym]{glossaries}

%Anpassung von Kopf- und Fußzeile
%beeinflusst die erste Seite des Kapitels
\usepackage[automark,headsepline]{scrlayer-scrpage}
%\input{resources/styles/header_footer}
\setlength{\parindent}{0em}

%\renewcommand\chapter{\thispagestyle{plain}}

%Auskommentieren für die Verkleinerung des vertikalen Abstandes eines neuen Kapitels
%\renewcommand*{\chapterheadstartvskip}{\vspace*{.25\baselineskip}}
%\renewcommand*{\chapterheadendvskip}{\vspace*{.25\baselineskip}}

%Zeilenabstand 1,5
\usepackage[onehalfspacing]{setspace}

%Verbesserte Darstellung der Buchstaben zueinander
\usepackage[stretch=10]{microtype}

%Deutsche Bezeichnungen für angezeigte Namen (z.B. Inhaltsverzeichnis etc.)
\usepackage[ngerman]{babel}

%Unterstützung von Umlauten und anderen Sonderzeichen (UTF-8)
\usepackage{lmodern}
\usepackage[utf8]{luainputenc}
\usepackage[T1]{fontenc}
\usepackage[ngerman]{babel}
\setlength{\emergencystretch}{1em}

%Einfachere Zitate
\usepackage{epigraph}

%Unterstützung der H Positionierung (keine automatische Verschiebung eingefügter Elemente)
\usepackage{float}

%Erlaubt Umbrüche innerhalb von Tabellen
\usepackage{tabularx}

%Erlaubt Seitenumbrüche innerhalb von Tabellen
\usepackage{longtable}

%Erlaubt die Darstellung von Sourcecode mit Highlighting
\usepackage{listings}

%Definition eigener Farben bei Nutzung eines selbst vergebene Namens
\usepackage[table,xcdraw]{xcolor}

%Vektorgrafiken
\usepackage{tikz}

%Grafiken (wie jpg, png, etc.)
\usepackage{graphicx}

%Grafiken von Text umlaufen lassen
\usepackage{wrapfig}

%Ermöglicht Verknüpfungen innerhalb des Dokumentes (e.g. for PDF), Links werden durch "hidelink" nicht explizit hervorgehoben
\usepackage[hidelinks,german]{hyperref}

\setlength{\headheight}{18pt}

%Zusätzliche Farben
\definecolor{darkgreen}{RGB}{0,100,0}
\setlength{\emergencystretch}{15pt}

%Anpassungen für das Abkürzungsverzeichnis
\newglossarystyle{dottedlocations}{%
	\renewcommand*{\glossaryentryfield}[5]{%
		\item[\glsentryitem{##1}\glstarget{##1}{##2}] \emph{##3}%
		\unskip\leaders\hbox to 2.9mm{\hss.}\hfill##5}%
	\renewcommand*{\glsgroupskip}{}%
}


\usepackage{graphicx}
\usepackage{subcaption}
\usepackage{float}
\usepackage{color}
\usepackage{listings}
\usepackage{eso-pic}

\lstset{literate=%
  {Ö}{{\"O}}1
  {Ä}{{\"A}}1
  {Ü}{{\"U}}1
  {ß}{{\ss}}1
  {ü}{{\"u}}1
  {ä}{{\"a}}1
  {ö}{{\"o}}1
}


% Neuer Befehl um Subsubsubkapitel schreiben zu können
\newcommand{\subsubsubsection}[1]{\paragraph{#1}\mbox{}\\}
\setcounter{secnumdepth}{4}
\setcounter{tocdepth}{4}

\begin{document}
\hypersetup{pageanchor=false}

\makeatletter

\renewcommand*{\maketitle}{
	\begin{titlepage}
		\newgeometry{left=3.8cm,right=2.5cm,top=4.6cm,bottom=2.5cm}
			\begingroup
				\fontsize{44pt}{46pt}\selectfont
				{\bfseries Gutachten }
			\endgroup

			\vskip 1.44cm

			\begingroup
			\fontsize{8pt}{6pt}\selectfont
			Sache
			\endgroup

			\vskip -0.02cm

			\begingroup
			\fontsize{16pt}{16pt}\selectfont
			Firma 1 / Firma 2
			\endgroup
			\vskip -0.1cm

			\noindent\rule{15cm}{0.4pt}

			\begingroup
			\fontsize{8pt}{6pt}\selectfont
			Auftraggeber
			\endgroup

			\vskip -0.02cm

			\begingroup
			\fontsize{16pt}{16pt}\selectfont
			Musterkanzlei
			\endgroup
			\vskip -0.1cm

			\noindent\rule{15cm}{0.4pt}

			\begingroup
			\fontsize{8pt}{6pt}\selectfont
			Ersteller
			\endgroup

			\vskip -0.02cm

			\begingroup
			\fontsize{16pt}{16pt}\selectfont
			Prof. Dr.-Ing. Mustermann
			\endgroup
			\vskip -0.1cm

			\noindent\rule{15cm}{0.4pt}
			
			\vskip 0.05cm

			\begingroup
			\fontsize{8pt}{6pt}\selectfont
			Auftragsdatum
			\endgroup

			\vskip -0.15cm

			\begingroup
			\fontsize{12pt}{14pt}\selectfont
				{Start - Stop}
			\endgroup
			\vskip -0.15cm

			\noindent\rule{15cm}{0.4pt}
		\restoregeometry
	\end{titlepage}
}
\makeatother


\pagenumbering{gobble}

% Titelblatt erzeugen
\maketitle
\newpage

\thispagestyle{empty}
\newpage
\hypersetup{pageanchor=true}
\pagenumbering{Roman}
\tableofcontents
\newpage
\listoffigures
\newpage

% Kapitel einfügen
\pagenumbering{arabic}
\section*{Auftragsspezifikation}
\label{sec:auftragsspezifikation}
Der Auftrag dieses....

\subsection{Auftragsformulierung}
\label{sec:auftragsformulierung}
Die Auftragsformulierung...

\subsection{Untersuchungszeitraum}
\label{sec:untersuchungszeitraum}
Als Basis für dieses Gutachten dienten Daten im Zeitraum vom ...

\subsection{Untersuchungsfragen und Sachverhaltsangaben}
\label{sec:untersuchungsfragen}
Folgende Fragen dienten als Basis für dieses Gutachten...


\newpage

\section{Zusammenfassung der Ermittlungsergebnisse}
\label{sec:ermittlungsergebnisse}
Allgemein lässt sich feststellen, dass...

Das hier ist ein \textit{Unterunterkapitel} \cite{Statista2018}

\newpage

\section{Untersuchtungsobjekte}
\label{sec:untersuchtungsobjekte}

Folgende Objekte dienten als Basis dieses Gutachtens...
\newpage

\section{Abschliessende Bemerkungen}
\label{sec:bemerkungen}

Abschliessend sollten noch folgende Punkte dediziert hervorgehoben werden:
\newpage

% Bibliografie im .bib Format einfügen
\bibliography{lit}

\end{document}
